\documentclass[a4paper,11pt]{article}
\usepackage{apacite}
\usepackage[spanish]{babel}
\usepackage[utf8]{inputenc}
\usepackage[dvips]{graphicx}
\DeclareGraphicsExtensions{.png}
\usepackage{multirow}
\usepackage{float}
\graphicspath{ {img/} }
\begin{document}
\title{CENTROS DE PROVEEDORES DE INFORMACIÓN EN EL ECUADOR}
\author{Jenny Arízaga Gamboa}
\date{\today}
\maketitle
\begin{bf}
URL del Repositorio:
\end{bf}
https://github.com/jennyarizagag/proyecto\_final
\begin{bf}
\begin{center}
RESUMEN \\
\end{center}
\end{bf}
El artículo se refiere a los ISP o Proveedor de Servicios de Internet (Del inglés Internet Service Provider),
es el término con el que se identifica a las compañías que proporcionan acceso a Internet, tanto a lo
s hogares como a otras empresas. En el Ecuador lleg\'o el internet en 1991 a trav\'es de la empresa Ecuanex,
en ese entonces, la mayoría de los usuarios de Internet lo utilizaban, sobre todo, para enviar un mensaje 
de manera instant\'anea, siendo su avance lento y paulatino debido al costo, hasta llegar a finales de l
a d\'ecada de los 90 en que se masificaron los Cibercaf\'es y el internet en las instituciones financieras, 
acad\'emicas y empresariales. Dada la acelerada evoluci\'on de las tecnolog\'ias, las barreras para el uso 
y masificaci\'on del Internet se han ido superando y pasamos de las tecnolog\'ias de banda estrecha a las 
tecnolog\'ias de Internet de banda ancha.\\   
\begin{bf}
Palabras claves:
\end{bf}
ISP, internet. \\
\begin{bf}
\begin{center}
INTRODUCCI\'ON\\
\end{center}
\end{bf}
La aplicaci\'on de las tecnolog\'ias de la informaci\'on y la comunicaci\'on en la actualidad han abarcado 
todos los aspectos de las empresas, porque estas empresas se han dado cuenta de la necesidad de aumentar 
la eficiencia y la rentabilidad.\\
Hace mucho tiempo, los seres humanos se vieron obligados a almacenar informaci\'on en su memoria para su 
uso futuro. Esto continu\'o durante mucho tiempo hasta la introducci\'on de mecanismos tecnol\'ogicos que 
eventualmente condujeron a la fabricaci\'on de computadoras, ahora las computadoras se utilizan en las 
organizaciones para almacenar un gran volumen de informaci\'on que se recupera y se accede a trav\'es de Internet.\\
El termino tecnolog\'ia de la informaci\'on y las comunicaciones se utiliza a menudo como un sin\'onimo extendido 
de tecnolog\'ia de la informaci\'on (TI), pero es un t\'ermino más espec\'ifico que enfatiza el papel de la 
comunicaci\'on y la integraci\'on de las telecomunicaciones (líneas telef\'onicas y señales inal\'ambricas), 
computadoras y software, middleware, almacenamiento y sistemas audiovisuales, que permiten a los usuarios 
acceder, almacenar, transmitir y manipular informaci\'on en \cite{Bunnell2000}.
\begin{bf}
\begin{center}
ESTADO DEL ARTE\\
\end{center}
\end{bf}
\begin{bf}   
ISP\\
\end{bf}
ISP o Proveedor de Servicios de Internet (Del inglés Internet Service Provider), es el t\'ermino con el que se 
identifica a las compañías que proporcionan acceso a Internet, tanto a los hogares como a otras empresas.\\
\begin{bf}
Proveedores de Internet\\ 
\end{bf}
Los proveedores de servicios en el Ecuador , el ente regulador es Agencia de Regulaci\'on y Control de las 
Telecomunicaciones ARCOTEL,un Proveedor de Servicio de Internet (Internet Service Provider – ISP) se encarga 
de conectar usuarios finales y negocios al Internet P\'ublico. Como se observa en la Figura 1, el n\'umero de ISPs 
ha crecido aproximadamente veinte y un veces a nivel nacional. Debido al alto número de proveedores existe una 
alta competencia en cuanto a precios, calidad, garantía de servicio, entre otros.\\
Entre las características de una Intranet se encuentran: Confidencialidad, 
Integridad, Autenticación, Verificación, Disponibilidad \cite{Dewit2001} y \cite{Costales2012}.
\begin{figure}[htb]
\centering
   \includegraphics[0cm,0cm][5cm,5cm]{figura1.png}
   \caption{Evoluci\'on de los ISP en Ecuador}
\label{Figura 1}
\end{figure}
\begin{bf}
\\
\\
Los Servicios que Ofrecen los ISP en el Ecuador\\
\end{bf} 
Acceso a Internet
Telefonía Fija
Dominios
Hosting
Housing.\\  
\begin{bf}
Como escoger correctamente un Proveedor de Servicios de Internet o ISP .\\
\end{bf}
Cobertura:
Antes de escoger un ISP hay que estudiar si ofrecen cobertura solo en grandes ciudades o en todo el ámbito nacional.\\
Ancho de banda:
¿Qué velocidad ofrece? Este es un criterio importante, ya que ese ancho de banda se comparte entre el número de usuarios.\\ 
Precio:
Varia según el ISP y los servicios elegidos.\\
Acceso:
Algunos ISP ofrecen un paquete por tiempo de conexión; es decir, no puede exceder el nº de horas contratado en un 
tiempo determinado. Muchos ISP ahora ofrecen pagar por la comunicación, pero puede llegar a ser más caro.\\
Soporte técnico:
Un servicio técnico de calidad, puede ser una gran ventaja a la hora de la resolución de incidencias.\\
Servicios adicionales:
Atención al cliente, espacio, cantidad de direcciones de email, etc.\\ 
\begin{bf}  
Análisis de Internet en Ecuador\\
\end{bf}    
Seg\'un estadísticas de la UIT el número de individuos usando Internet es de aproximadamente 4.000 millones a nivel mundial. Como 
observamos en la figura 1, en el 2001 eran aproximadamente 500 millones creciendo en un 800\% hasta el 2019.
Uno de los terminos mas utilizados es Banda Ancha en el Ecuador se lo define como el Ancho de banda suministrado a un usuario 
mediante una velocidad de transmisión de bajada (proveedor hacia usuario) mínima efectiva igual o superior a 256  Kbps 
y una velocidad de transmisión de subida (usuario hacia proveedor) mínima efectiva igual o superior a 128 Kbps para 
cualquier aplicación \cite{Dewit2001}.
\begin{figure}[htb]
\centering
   \includegraphics[0cm,0cm][8cm,5cm]{figura2.png}
   \caption{Numero global de individuos que usan Internet por cada 100 habitantes}
\label{Figura 1}
\end{figure}
\\
\\
Seg\'un estadisticas de la Internet World Stats el porcentaje de penetraci\'on de internet en el Ecuador es de aproximadamente
del 80\% a Julio del 2020, ubicandose en el cuarto puesto (Ver Cuadro 1)
\begin{table}[h]
\begin{center}
\begin{tabular}{| c | c | c | c | c |}
\hline
\multicolumn{5}{ |c| }{ Porcentaje de Penetraci\'on de Internet} \\ \hline Paises & Poblaci\'on 2020 & \% Poblaci\'on & Usuarios Internet 2020 & \% Penetraci\'on \\ \hline Argentina & 44.688 &
 10.48 & 41.586 & 93.06 \\ \hline Paraguay & 6.896 & 1.62 & 6.177 & 89.57 \\ \hline Uruguay & 3.469 & 0.81 & 3.059 & 88.19 \\ \hline Ecuador & 16.863 & 3.95 & 13.476 & 79.92 \\ \hline
\end{tabular}
\caption {Tabla de porcentaje de penetraci\'on del Internet en Am\'erica Latina}
\end{center}
\end{table}
\\
Para el caso de Ecuador el servicio de acceso a Internet representa uno de los servicios con mayor demanda y crecimiento debido
fundamentalmente a la cantidad de contenido generado y compartido a trav\'es de la red, el desarrollo de aplicaciones y el acceso
a redes sociales, particular interes reviste para el estado Ecuatoriano el promover el aumento penetraci\'on de servicios de banda ancha fija y m\'ovil tal como se indica en el Plan Nacional de Telecomunicaciones y Tecnolog\'ias de Informaci\'on y Comunicaci\'on.\\
\\
la formula utilizada para los calculos, es la de Euler que demostro que la serie $\sum\limits_{n=1}^\infty\frac{1}{n^2}$ converge, pero ademas que:
\begin{equation}
\sum\nolimits_{n=1}^\infty\frac{1}{n^2}=\frac{\pi^2}{6}
\end{equation} 
\begin{bf}
CONCLUSI\'ON\\
\end{bf}
El uso de Internet se ha desarrollado desde lo militar, academico, industrial hasta convertirse en indespensable, 
gracias a la evoluci\'on tecnol\'ogica que ha permitido en la actualidad que podamos acceder a una infinidad 
de informaci\'on a traves de varias tecnol\'ogias fijas y m\'oviles, los proveedores de servicio (ISP) en el 
Ecuador han aportado con tecnolog\'ia y recurso humano permitiendo que gran parte de la poblaci\'on acceda a 
Internet, colaborando con una serie de servicios de valor agregado para cumplir con las politicas de estado 
que declara a Internet como un derecho b\'asico para todos los Ecuatorianos.\\
\\
\bibliographystyle{apacite}
\bibliography{M6_Arizaga_Gamboa_Jenny}
\end{document}
